\chapter{Basics}
Zeilenumbrüche mit \textbackslash \textbackslash \\
\textbf{fetter Text} mit \textbackslash textbf (Str+B) \\
\textit{kursiver Text} mit \textbackslash textit (Str+I) 
\par Neuer Absatz mit \textbackslash par
\par \textbackslash chapter\{\}, \textbackslash section\{\}, \textbackslash subsection\{\}, \textbackslash subsubsection\{\} erstellt Abschnitte, Kapitel 


\section{Abschnitt}
\subsection{Unterabschnitt}
\subsubsection{Unterunterabschnitt}


\section{Quellenangaben}
\subsection{Im Text}
\textbackslash cite[Vgl.][]\{Name der Quelle im Verzeichnis\}
% glossary entries with gls
\par \gls{LLM}s have been heavily researched for the past 10 years (\cite[Cf.][]{sunSurveyLargeLanguagea})

\subsection{Referenzierung von Titel oder Autor}
\textbackslash citeauthor oder \textbackslash citetitle
\par Though there is a study published by \citeauthor{sunSurveyLargeLanguagea} which examines how \gls{LLM}s work: \citetitle{sunSurveyLargeLanguagea}.

\section{Aufzählungen}
enumerate bzw. itemize
\begin{enumerate}
    \item Eins
    \item Zwei
    \item Drei
\end{enumerate}

\begin{itemize}
    \item Stichpunkt
    \item Stichpunkt
    \item Stichpunkt
\end{itemize}

\section{Fußnoten}

Only songs with the biggest appeal to the public will be analysed. The song selection is based on the \textit{Billboard Top 100 Weekly Charts}\footnote{\url{https://www.billboard.com/charts/hot-100}} which feature the most played songs during one week. 

\section{Bilder}
Mit figure. Beim Kompilieren werden die Bilder automatisch an die beste stelle geschoben, also tauchen nicht immer direkt unterm Text auf. 
Dies kann unterbunden werden mit [h] hinter figure.
\begin{figure}[h]
    \centering
    \includegraphics[width=0.8\textwidth]{../img/htw_logo_pride}
    \caption{HTW Logo}
    \label{fig:enter-label}
\end{figure}

